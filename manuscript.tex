\documentclass{infor}

\volume{0}
\issue{0}
\pubyear{202X}
\articletype{research-article}
\doi{0000}

\newtheorem{thm}{Theorem}
\newtheorem{remark}{Remark}
\newtheorem{lemma}{Lemma}
\newtheorem{cor}{Corollary}

\theoremstyle{remark}
\newtheorem{defin}{Definition}

\hyphenation{de-si-de-rium}


\begin{document}
\begin{frontmatter}
\pretitle{Research Article}

\title{Comparative analysis between CPT-TODIM and TODIM by assessment of social sustainable performance for G7 countries}

% Short title suggestion:
% \title{Comp. anlys. b/n CPT-TODIM & TODIM by assmt. of social sust. perf. for G7 countries}

\author[a]{\inits{V.}\fnms{Vaishnudebi} \snm{Dutta}\thanksref{c1}\ead[label=e1]{vaishnudebi.dutta7@gmail.com}\bio{bio1}}
\author[a]{\inits{S.}\fnms{Subhomoy} \snm{Haldar}\ead[label=e2]{subhomoyhaldar@gmail.com}\bio{bio2}}
\thankstext[type=corresp,id=c1]{Corresponding author}
%\thankstext[id=f1]{Simple footnote.}
\address[a]{Department of Mathematics, \institution{Birla Institute of Technology, Mesra}, \cny{Jharkhand - 835215, India}}

\begin{abstract}
Social sustainability objectives within the framework of sustainable development goals are critical aspects of balanced societies. Governments must constantly assess their performance to accomplish social sustainability goals. This project evaluates the performance of seven industrialised nations: members of the Group of Seven (G7) - one of the most prominent international economic groups, in a very streamlined manner. We use the rank sum weighting approach to quantify the subjectivity of experts’ judgments. TODIM and CPT-TODIM are used to assess the social sustainability performance and conduct a comparison. We perform sensitivity analysis to evaluate and contrast the two MCDM algorithms used.
\end{abstract}

\begin{keywords}
\kwd{Social Sustainability}
\kwd{Multi-Criteria Decision-Making}
\kwd{Prospect Theory}
\kwd{TODIM}
\kwd{CPT-TODIM}
\kwd{Sensitivity Analysis}
\end{keywords}

\end{frontmatter}

\section{Introduction}

\subsection{On Sustainability}

Among the several approaches to sustainability and sustainable development, social sustainability is the least defined and least understood. Humanity’s mindset at present is that, nature is an infinite resource which can be profited off of without paying regard to the consequences. This is obviously not the case, and several sectors of economy are already developing sustainable applications. Apart from that, the public sector is enacting mandatory rules for sustainable practices, and sustainability is being viewed as an integral characteristic of an ideal society \citep{vassallo2021sustainability, chowdhury2021review, swann2021linking}. In this framework, all sectors are taking responsibility for providing a habitable environment for future generations. All relevant sectors should adopt their most appropriate and promising policies for deliberately utilising resources, acting in a people-centric manner, implementing economic, social, and environmental policies, and ultimately, formulating policies on sustainable-oriented platforms \citep{weber2021environmental, tzouramani2020assessing, maynard2020environmental, van2020inclusive}.

The notion of environmental sustainability has received a lot of attention since the 1960s, when widespread worries about environmental degradation arose. John Elkington, an environmentalist and economist, coined the phrase “triple bottom line” in \cite{elkington1997triple}, and it has since become a catchphrase for a type of corporate reporting that considers environmental, social, and economic problems. Environmental policy has an influence on the advancement of greener technologies. We can see how Kemp in his book \cite{kemp1997environmental} shed light on the relationship between clean technology and environmental policy in 90s. However, amid short allusions to sustainability, social sustainability has been disregarded for roughly two decades. This is despite the notion of sustainable development containing a strong social mandate at its inception. As a consequence of the widespread failure of this approach to produce significant change, increased interest in the notion of social sustainability has emerged \citep{vallance2011social}. There are several articles dedicated just to clarify the theoretical concepts of social sustainability \citep{dempsey2011social, eizenberg2017social, mckenzie2004social, magis2010community, polese2000social}.

\subsection{Nations and their Role in Sustainability}

Topics like social equality, livability, health equity, community development, social capital, and social support are all included under the umbrella of social sustainability. When viewed from a novel perspective, all areas of sustainability, including ecological, economic, political, and cultural sustainability, are social in nature \cite{schonborn2019social}. These social sustainability domains are all reliant on the ecological domain, which is defined as human embeddedness in the environment and includes all human activities \citep{james2014urban}. Social sustainability is one of the three pillars of sustainable development that has unfortunately received the least attention throughout the world. As much as the restricted scope allows, social sustainability is defined as the notion of social capital, social exclusion, and fundamental social requirements, such as health, housing, and nourishment. It is also seen as a way to emphasise the importance of societal ideals like culture, equality, and social justice \citep{koning2001social}. Countries strive to enhance their performance on attaining sustainable development objectives in the three pillars of social, economic, and environmental goals in a world full of unpredictability and dynamicity \citep{precup2020model}. Countries continually analyse their development and present accomplishment level in relation to each linked indicator for this reason. Developed and developing nations both devote a significant portion of their annual budgets to improving their performance in reaching the objectives in contrast to their prior position and that of other countries. Such assessments can be carried out worldwide within well-known groups of nations, such as the United States and OECD (Organisation for Economic Co-operation and Development) or the Group of Seven (G7). Canada, France, Germany, Italy, Japan, the United Kingdom, and the United States of America are members of the G7. The G7 countries are perceived as developed countries with substantial global net worth.

The MCDM method, or Multi Criteria Decision Making, is used extensively to build decision-making tools for issues pertaining to making difficult choices in a variety of situations. Most of these studies \citep{kumar2020mcdm, khokhar2020evaluating, perez2021mcdm, torkayesh2021comparative, kumar2019development, anand2017evaluation, papathanasiou2016decision, mani2014supplier} concentrated on analysing social sustainability at the sector or city level, however there are just a few publications that employ MCDM to solve social sustainability issues. Our research article demonstrates a streamlined method of comparing the present state and progress of the G7 countries in terms of the social sustainability. Essential criteria or indicators connected to social sustainability in many areas, such as healthcare, education, and home economics, have been identified for this purpose. The criteria and dataset used in this paper are taken from \cite{torkayesh2021comparative} which was originally taken from \href{https://www.oecd-ilibrary.org/}{OECD ilibrary}. We undertake a genuine case study for economically developed nations in the G7 organisation to illustrate how the proposed decision-making may be efficient and relevant as well. Previous research in the literature on social sustainability has utilised a variety of descriptive approaches, such as surveys, questionnaires, conceptual modelling, and so on, according to the nature of social issues \citep{abousaeidi2020developing, navas2020colombian, d2019social, hussain2018exploration, mani2016social}.

Our study shows the application of TODIM (TOmada de Decisão Interativa e Multicritério - an acronym in Portuguese for Interactive Multi-criteria Decision Making) \citep{gomes2009application} and its derivative CPT-TODIM (Cumulative Prospect Theory TODIM) \citep{tian2019extended} on the G7 social sustainability dataset. We have implemented Rank-Sum Weight method \citep{einhorn1977simple, stillwell1981comparison, jia1998attribute, hu2007selection} owing to the ease of calculations and to provide flexibility to the policy or decision makers with the ranking of the criteria according to their importance.

\section{Methodology}
\subsection{Weighting through Rank Sum Weight
Method}

This subjective technique generates weights purely based on the considerations or opinions of the decision makers \citep{zoraghi2013fuzzy}. It may be simpler to rank order the importance of criteria than to explain alternative inexact weights, such as bounded weights. For example, time limitations, the nature of the criterion, a lack of knowledge, imprecise or incomplete information, and the decision maker’s limited attention or information processing abilities are all factors to consider. Because a group of decision-makers may not agree on a set of exact weights, it may be fair to presume agreement on a weight ranking instead, as mentioned in the article \cite{roszkowska2013rank}. This rank order weight method involves two steps: first, ranking the criteria by importance, and then, using the formula, weighting the criteria. \cite{stillwell1981comparison} devised the rank-sum weight technique, which we employed in this study. Individual ranks are normalised using the rank-sum (RS) approach, which divides the total of the rankings. The weights are calculated using the following formula:

\begin{equation}
\label{eqn:rankSumWeights}
w_{j}\left( \text{RS} \right) = \frac{2\left(n + 1 - r_j \right)}{n\left( n + 1 \right)}
\end{equation}

Here \(r_j\) is the rank of the \(j\)th criteria,
\(j = 1, 2, \ldots, n\).

If there are several decision makers, the rankings can be established through deliberation. Alternatively, the final rank values can be calculated by adding the individual decision makers' ranks and averaging them. It is not a strict requirement for the `ranks' to be integers, but it makes the procedure easier to comprehend.
\subsection{Classical TODIM}

The TODIM method is based on Prospect Theory \citep{kahneman2013prospect,gomes2009application, wang2018todim}. Therefore, gains and losses are always calculated relative to a reference point. As a result, while this method acknowledges the possibility of decision makers, it does not take into account their actual participation. A TODIM application would go through the following steps in algorithmic form:

\begin{steps}

\item We will use the initial evaluation matrix and normalize the ratings and weights using the formula:

\begin{equation}
\label{eqn:normalizationFormula}
P_{ij} = \left\{ \begin{matrix}
\frac{x_{ij}}{\sum_{k = 1}^{m}x_{kj}} & \text{if}\ j \in J_{1} \\
\frac{\frac{1}{x_{ij}}}{\sum_{k = 1}^{m}\frac{1}{x_{kj}}} & \text{if}\ j \in J_{2} \\
\end{matrix} \right.
\end{equation}
\item Now we calculate the weighting factor or trade-off rate between the reference criteria \(r\) and the generic criteria \(c\). Here, \(w_{r}\) determines the most relevant reference criterion for the decision maker. Often, is it the maximum weight. In general, any criterion can be used as the reference criterion, and this decision has no effect on the final findings. We use the formula from equation \ref{eqn:weightScaling} to compute the weighting factor \(w_{rc}\):
\begin{equation}
\label{eqn:weightScaling}
w_{rc} = \ \frac{w_{c}}{w_{r}}
\end{equation}
Where \(w_{r} = \max{\{ w_{c}\ |c = 1,2, \ldots, n\}}\) and
\(i=1,2\ldots,m\); \(j = 1, 2\ldots, n\).

\item For calculating the dominance degree we need to check the contribution of each criteria where $\varrho_{c}$ is the contribution of criterion $c$ to the function $\delta\left(A_{i,j}\right)$ and $\theta$ is the attenuation factor. A higher value of $\theta$ means we are risk averse.

\begin{equation}
\label{eqn:dominanceDegrees}
\varrho_{c}\left( A_{i},\ A_{j} \right) = \left\{ \begin{matrix}
\sqrt{\frac{(P_{ic} - P_{jc})w_{rc}}{\sum_{c = 1}^{n}w_{rc}}},& \text{if\ \ }P_{ic} - P_{jc} > \ 0 \\
0,& \text{if\ \ }P_{ic} - P_{jc} = 0 \\
 - \frac{1}{\theta}\sqrt{\frac{\left( \sum_{c = 1}^{n}w_{rc} \right)\left( P_{jc} - P_{ic} \right)}{w_{rc}}},& \text{if\ \ }P_{ic} - P_{jc} < 0 \\
\end{matrix} \right.
\end{equation}

We can combine all contributions to get the dominance degrees for the measurement of dominance \(\delta\left(A_{i},A_{j} \right)\) in the following manner:
\begin{equation}\label{eqn:deltaSums}
\delta\left( A_{i},\ A_{j} \right) = \ \sum_{c = 1}^{n}{\varrho_{c}\left( A_{i},\ A_{j} \right)}
\end{equation}
where \(i,j  = 1, 2\ldots, m\); \(c = 1, 2\ldots, n\).

\item Finally, we compute the values of \(\xi_{i}\) which are the normalised global performances of
alternatives compared to others:

\begin{equation}\label{eqn:xiValues}
\xi_{i} = \ \frac{\sum_{j = 1}^{n}{\delta\left( A_{i},A_{j} \right) - \min_{i}{\sum_{j = 1}^{n}{\delta\left( A_{i},\ A_{j} \right)}}}}{\max_{i}{\sum_{j = 1}^{n}{\delta\left( A_{i},A_{j} \right) - \min_{i}{\sum_{j = 1}^{n}{\delta\left( A_{i},\ A_{j} \right)}}}}}
\end{equation}

where \(i\  = 1,\ 2,\ldots,\ n\).

A larger value of \(\xi_{i}\) signifies a better ranking.

\end{steps}

\subsection{ CPT-TODIM (Cumulative Prospect Theory-TODIM)}

CPT-TODIM is an improved version of TODIM that provides a better, more nuanced method for determining attribute weights. It also comprehensively expresses the true perceptions of a decision maker’s gains or losses.

The algorithm stated by \cite{tian2019extended} is as follows:

\begin{steps}

\item We now solve for the transformed probability of the alternatives \(A_{i}\) to \(A_{k}\) where \(k \in M\) and \(k \neq i\). When \(x_{ij} - x_{kj} \geq 0\), the transformed probability weight can be solved using:
\begin{equation}\label{eqn:phiPlusIKJ}
\varphi_{ikj}^{+}\left( w_{j} \right) = \frac{w_{j}^{\vartheta}}{\left( w_{j}^{\vartheta} + \left( 1 - w_{j} \right)^{\vartheta} \right)^{\frac{1}{\vartheta}}}
\end{equation}
Else, when \(x_{ij} - x_{kj} < 0\) , we can calculate probability weight using:
\begin{equation}\label{eqn:phiMinusIKJ}
\varphi_{ikj}^{-}\left( w_{j} \right) = \frac{w_{j}^{\mu}}{\left( w_{j}^{\mu} + \left( 1 - w_{j} \right)^{\mu} \right)^{\frac{1}{\mu}}}
\end{equation}

Where, \(\vartheta\) and \(\mu\) are the
parameters that describe the curvature of the weighting function and
represent the variations in declining sensitivity in the domain of gains
and losses.

The relative weight is then calculated \(\varphi_{ikj^{*}}\) for the alternative \(A_{i}\) to the alternative \(A_{k}\) given by the
following formula:
\begin{equation}\label{eqn:phiRelative}
\varphi_{ikj}^{*} = \frac{\varphi_{ikj(w_{j})}}{\varphi_{ikr}(w_{r})}
\end{equation}

where \(r,j \in M\), \(\forall(i,k)\). We have
\(\varphi_{ikj}(w_{j})\) which represents the transformed weight
of the \(j\)th criteria for alternative \(A_{i}\) and
\(\varphi_{ikj}(w_{r})\) represents transformed weight of
reference attribute for the alternative \(A_{i}\) to \(A_{k}\) where
\begin{equation}\label{eqn:maxPhiWeight}
\varphi_{ikr}\left( w_{r} \right) = \max\left( \varphi_{ikj}\left( w_{j} \right)\ |\ j \in 1,2,\ldots,m \right)
\end{equation}

\item Relative prospect dominance of the alternative \(A_{i}\) over \(A_{k}\) under the attribute \(j\) can be calculated by using the formula:

\begin{equation}\label{eqn:tauJStar}
\tau_{j^{*}}\ \left( A_{i},\ A_{k} \right) = 
\begin{cases}
\frac{\varphi_{ikj^{*}}\left( x_{ij} - x_{kj} \right)^{\alpha}}{\sum_{j^{*} = 1}^{m}{\varphi_{ikj^{*}}\ }}, & \text{if }x_{ij} > x_{kj} \\
0,  & \text{if }x_{ij} - x_{kj} = 0 \\
\frac{- \sigma(\sum_{j^{*} = 1}^{m}{\varphi_{ikj^{*}})\left( x_{kj} - x_{ij} \right)^{\beta}}}{\varphi_{ikj^{*}}},&\text{if }x_{ij} < x_{kj} \\
\end{cases}
\end{equation}
Here, \(\alpha\) and \(\beta\) are the dimensions of DM's risk attitudes and are interpreted as preference degrees in the domains of gain and loss, respectively. The parameter \(\sigma\) is a loss aversion parameter
that is more sensitive to loss than gain. As a result, the value function represents the various risk attitudes toward profits and losses.

\item The sum of relative prospect dominance of the
alternatives can be stated as:
\begin{equation}\label{eqn:deltaSumCPTTODIM}
\delta\left( A_{i},A_{k} \right) = \sum_{j^{*} = 1}^{m}{\tau_{j^{*}}(A_{i},A_{k})}
\end{equation}
Then we calculate the overall value using \(\xi_{i}\) stated in TODIM algorithm (Equation \ref{eqn:xiValues}), then rank the values of prospect dominance \(\xi\left( A_{i} \right)\) where \(i \in 1, 2, \ldots, n\). The bigger the values are for \(\xi\left( A_{i} \right)\), the better ranking \(A_{i}\) receives. This is exactly the same as in TODIM's last step.
\end{steps}

\section{Case Study}

\subsection{Criteria Selection}

In order to evaluate the performance of the alternatives, we choose various criteria for assessment of social sustainability performance. They are sourced from \cite{torkayesh2021comparative} and are reproduced in \autoref{Table 1}.

\begin{table}[!htp]\centering
\caption{Criteria for Social Sustainability Assessment}\label{Table 1}
\scriptsize
\begin{tabular}{lrrrrrr}\toprule
\textbf{Indicator} &\textbf{Name} &\textbf{Unit} &\textbf{Ideally} &\textbf{Rank} &\textbf{Weight}\\ \cmidrule{1-6}
C1 &The average wage &US Dollar &Higher &11 &0.038095\\%\cmidrule{1-5}
C2 &The employment rate & Percent of the working age population &Higher &7 &0.07619 \\ %\cmidrule{1-5}
C3 &Income inequality &Ratio &Lower &14 &0.009524\\ %\cmidrule{1-5}
C4 &Labor force &Thousand persons &Higher &1 &0.133333\\ %\midrule
C5 &Poverty gap &Ratio &Lower &10 &0.047619\\
C6 &Poverty rate &Ratio &Lower &9 &0.057143\\
C7 &Working hours &Hours/worker &Higher &8 &0.066667\\
C8 &Women in politics &Percentage &Higher &5 &0.095238\\
C9 &Population density &Ratio &Lower &2 &0.12381\\
C10 &Adult education level & Percent of 25-64 year-old &Higher &6 &0.085714\\
C11 &Spending on tertiary education & Percent of education spending &Higher &4 &0.104762\\
C12 &International student mobility & Percent of students enrolled &Higher &3 &0.114286\\
C13 &Tertiary graduation rate & Percent of the same level &Higher &13 &0.019048\\
C14 &Social spending & of GDP &Higher &12 &0.028571  \\
\bottomrule
\end{tabular}
\end{table}

\subsection{Application of TODIM}

After calculating the weights of the criteria, we will use TODIM to rank the alternative G7 countries, as previously discussed. The evaluation matrix is the same for both algorithms in the paper, and we dealt with a crisp dataset as follows:
\begin{steps}
\item The initial evaluation matrix for social sustainability
performance for G7 countries are represented below in \autoref{Table 2}:

\begin{table}[!htp]\centering
\caption{Initial Evaluation Matrix}\label{Table 2}
\scriptsize
\begin{tabular}{lrrrrrrrr}\toprule

& CA & FR & DE & IT & JP & UK & USA\\ \toprule

C1 & 53198.17 & 46480.62 & 53637.80 & 39189.37 & 38617.47 & 47226.09 &
65835.58\\
C2 & 64.73 & 66.02 & 76.09 & 59.07 & 77.95 & 75.61 &
62.56\\
C3 & 0.31 & 0.29 & 0.28 & 0.33 & 0.33 & 0.35 & 0.39\\
C4 & 20199.55 & 29682.22 & 43769.63 & 25941.40 & 68863.34 & 33964.07 &
163538.70\\
C5 & 0.30 & 0.25 & 0.25 & 0.40 & 0.33 & 0.34 & 0.38\\
C6 & 0.12 & 0.08 & 0.10 & 0.13 & 0.15 & 0.11 & 0.17\\
C7 & 1670 & 1505 & 1386.1 & 1717.8 & 1644 & 1538 & 1779\\
C8 & 51.7 & 52.9 & 33.3 & 27.8 & 15.8 & 30.8 & 16.7\\
C9 & 4 & 122 & 237 & 205 & 347 & 275 & 36\\
C10 & 57.88 & 36.89 & 29.06 & 19.32 & 51.92 & 45.78 &
47.43\\
C11 & 49.052 & 77.838 & 82.723 & 61.715 & 32.416 & 24.991 &
35.205\\
C12 & 12.917 & 10.201 & 8.373 & 5.311 & 4.265 & 17.918 &
5.18\\
C13 & 54.40 & 54.31 & 49.33 & 56.07 & 36.87 & 54.47 &
55.41\\
C14 & 20.89 & 31.68 & 24.76 & 25.36 & 23.51 & 24.49 &
30.02\\
\bottomrule
\end{tabular}
\end{table}
\item Normalized matrix is then calculated and shown in \autoref{Table 3} using the formula given in equation \eqref{eqn:normalizationFormula}.
\begin{table}[!htp]\centering
\caption{Normalized Evaluation Matrix}\label{Table 3}
\scriptsize
\begin{tabular}{lrrrrrrrr}\toprule
\textbf{} &\textbf{CA} &\textbf{FR} &\textbf{DE} &\textbf{IT} &\textbf{JP} &\textbf{UK} &\textbf{USA} \\\cmidrule{1-8}
\textbf{C1} &0.154563 &0.135045 &0.15584 &0.113861 &0.1122 &0.137211 &0.19128 \\
\textbf{C2} &0.134286 &0.136962 &0.157853 &0.122544 &0.161712 &0.156857 &0.129784 \\
\textbf{C3} &0.148467 &0.158707 &0.164375 &0.139469 &0.139469 &0.1315 &0.118013 \\
\textbf{C4} &0.052336 &0.076905 &0.113405 &0.067213 &0.178421 &0.087999 &0.42372 \\
\textbf{C5} &0.148568 &0.178282 &0.178282 &0.111426 &0.135062 &0.13109 &0.117291 \\
\textbf{C6} &0.138507 &0.20776 &0.166208 &0.127852 &0.110805 &0.151098 &0.097769 \\
\textbf{C7} &0.148578 &0.133898 &0.12332 &0.152831 &0.146265 &0.136834 &0.158275 \\
\textbf{C8} &0.225764 &0.231004 &0.145415 &0.121397 &0.068996 &0.134498 &0.072926 \\
\textbf{C9} &0.828939 &0.027178 &0.013991 &0.016174 &0.009555 &0.012057 &0.092104 \\
\textbf{C10} &0.200777 &0.127966 &0.100805 &0.067018 &0.180103 &0.158804 &0.164528 \\
\textbf{C11} &0.13478 &0.213876 &0.227298 &0.169575 &0.08907 &0.068668 &0.096733 \\
\textbf{C12} &0.201309 &0.158981 &0.130492 &0.082771 &0.066469 &0.279249 &0.080729 \\
\textbf{C13} &0.150751 &0.150502 &0.136701 &0.155379 &0.102173 &0.150945 &0.15355 \\
\textbf{C14} &0.1156 &0.175309 &0.137015 &0.140335 &0.130098 &0.135521 &0.166123 \\
\bottomrule
\end{tabular}
\end{table}
\item Weighting factor or the normalized weight is then calculated and the
results are given below in \autoref{Table 4} :

\begin{table}[!htp]\centering
\caption{Weighting Factor}\label{Table 4}
\scriptsize
\begin{tabular}{lrr}\toprule

Criteria & Weight\tabularnewline \toprule
C1 & 0.285714\\
C2 & 0.571429\\
C3 & 0.071429\\
C4 & 1.000000\\
C5 & 0.357143\\
C6 & 0.428571\\
C7 & 0.500000\\
C8 & 0.714286\\
C9 & 0.928571\\
C10 & 0.642857\\
C11 & 0.785714\\
C12 & 0.857143\\
C13 & 0.142857\\
C14 & 0.214286\\
\bottomrule
\end{tabular}
\end{table}

The dominance degree is then calculated using the value
of attenuation factor \(\theta = 2.5\) and the results are noted down
below in \autoref{Table 5}:

\begin{table}[!htp]\centering
\caption{Dominance Degrees}\label{Table 5}
\scriptsize
\begin{tabular}{lrrrrrrrr}\toprule
\textbf{} &\textbf{CA} &\textbf{FR} &\textbf{DE} &\textbf{IT} &\textbf{JP} &\textbf{UK} &\textbf{USA} \\\toprule
\textbf{CA} &0 &-1.913036 &-1.754999 &-0.24885 &-0.080357 &-0.667342 &-1.116732 \\
\textbf{FR} &-1.813196 &0 &-0.80545 &0.223884 &-0.444036 &-0.793794 &-1.563419 \\
\textbf{DE} &-2.377772 &-2.055961 &0 &-0.309479 &-0.476611 &-0.940191 &-2.269086 \\
\textbf{IT} &-3.678201 &-3.965399 &-3.294145 &0 &-1.050064 &-2.150764 &-2.489952 \\
\textbf{JP} &-4.257101 &-4.134444 &-3.546336 &-1.918023 &0 &-2.143187 &-2.895549 \\
\textbf{UK} &-2.973374 &-2.685963 &-2.368819 &-0.910031 &-1.063867 &0 &-2.167227 \\
\textbf{USA} &-3.538379 &-2.998469 &-2.567817 &-1.139834 &-0.948707 &-1.6577 &0 \\
\bottomrule
\end{tabular}
\end{table}

\item The ratings are then calculated from the normalized
dominance degrees. The values of \(\delta\left( A_{i},A_{j} \right)\)
are given in \autoref{Table 6}.

\begin{table}[!htp]\centering
\caption{Values of \(\delta \)}\label{Table 6} 
\scriptsize
\begin{tabular}{lrr}\toprule
\textbf{} &\textbf{Sum} \\ \cmidrule{1-2}
\textbf{CA} &-5.781315 \\
\textbf{FR} &-5.196011 \\
\textbf{DE} &-8.4291 \\
\textbf{IT} &-16.628526 \\
\textbf{JP} &-18.89464 \\
\textbf{UK} &-12.16928 \\
\textbf{USA} &-12.850907 \\
& \\\cmidrule{1-2}
Minimum &-18.89464 \\
Maximum &-5.196011 \\
\bottomrule
\end{tabular}
\end{table}

The \(\xi_{i}\) values in \autoref{Table 7} are then calculated
using the formula \eqref{eqn:xiValues}.

%\begin{longtable}[]{@{}ll@{}}
%\caption{Rating values \(\mathbf{\xi}\)}\label{Table 7}\\
\begin{table}[!htp]\centering
\caption{Rating values \(\mathbf{\xi}\)}\label{Table 7} 
\scriptsize
\begin{tabular}{lrr}\toprule
& \(\mathbf{\xi}_{\mathbf{i}}\)\\ \toprule

\textbf{CA} & 0.957273\\
\textbf{FR} & 1\\
\textbf{DE} & 0.763984\\
\textbf{IT} & 0.165426\\
\textbf{JP} & 0\\
\textbf{UK} & 0.490951\\
\textbf{USA} & 0.441193\\
\bottomrule
\end{tabular}
\end{table}
\end{steps}
As a result, when we rank the G7 countries using standard TODIM, we get the ranking as follows:
\begin{equation}
\text{FR} > \text{CA} > \text{DE} > \text{UK} > \text{USA} > \text{IT} > \text{JP}
\end{equation}
\subsection{Application of CPT-TODIM}

The regular TODIM algorithm considers the relative significance of attributes, but it neither provides an appropriate method for determining attribute weights nor expresses the true perceptions of DM gains or losses comprehensively. Such perceptions are captured by the value function of CPT \citep{tian2019extended}. By incorporating the transformed weighting function and value function of the prominent CPT into CPT-TODIM, we can make the method more acceptable for decision-making
environment as well as improve the accuracy of decisions for DMs.
\begin{steps}
\item Normalized matrix as shown in \autoref{Table 3} and the
normalized weights as shown in \autoref{Table 4} are same as that of TODIM.

\item The sum of relative prospect dominance of the
alternatives is then calculated using the parametric values
\(\alpha = 0.88,\ \beta = 0.88,\ \vartheta = 0.61,\ \mu = 0.69\) and
\(\sigma = 2.25\) whose results are shown in \autoref{Table 8}:

%\begin{longtable}[]{@{}ll@{}}
%\caption{Sum of the calculated dominance degree}\label{Table 8}\\
\begin{table}[!htp]\centering
\caption{Sum of the Calculated Dominance Degree}\label{Table 8} 
\scriptsize
\begin{tabular}{lrr}\toprule
& \textbf{Sum of} \(\tau\)\\\toprule
\
\textbf{CA} & -58.856412\\
\textbf{FR} & -55.494989\\
\textbf{DE} & -77.989684\\
\textbf{IT} & -131.312161\\
\textbf{JP} & -153.753302\\
\textbf{UK} & -98.104911\\
\textbf{USA} & -120.010459\\
&\\
\cmidrule{1-2}
Minimum & -153.753302\\
Maximum & -55.494989\\
\bottomrule
\end{tabular}
\end{table}

\item The calculated ratings shown in \autoref{Table 9} are used to rank
the alternatives.

%\begin{longtable}[]{@{}ll@{}}
%\caption{Ratings for CPT-TODIM}\label{Table 9}\\
\begin{table}[!htp]\centering
\caption{Ratings for CPT-TODIM}\label{Table 9} 
\scriptsize
\begin{tabular}{lrr}\toprule
& \textbf{Rating}\\ \toprule
\textbf{CA} & 0.96579\\
\textbf{FR} & 1\\
\textbf{DE} & 0.771066\\
\textbf{IT} & 0.228389\\
\textbf{JP} & 0\\
\textbf{UK} & 0.566348\\
\textbf{USA} & 0.34341\\
\bottomrule
\end{tabular}
\end{table}

\end{steps}
Similarly, when we rank the G7 countries using CPT-TODIM, as expected from the article by \cite{tian2019extended} we get same rank as that of regular TODIM as follows:
\begin{equation}
\text{FR} > \text{CA} > \text{DE} > \text{UK} > \text{USA} > \text{IT} > \text{JP}
\end{equation}

\section{Sensitivity Analysis}
\subsection{Weight Shuffling}
Sensitivity analyses for both TODIM and CPT-TODIM were performed, revealing the impact and stability of each technique. The rank sum weight approach is used in this study to generate the criterion weights. Parameters of both the algorithms differ and may react differently because of changing their values. As a result, in order to evaluate both methodologies, we limited our study to changes in criterion weights alone.

We use heat maps to represent sensitivity, as studied by \cite{pryke2007heatmap}. For each test, we have 1000 data points. Bar or line graphs cannot adequately summarise this enormous quantity of data. As a result, heat maps are chosen as an efficient technique for displaying the crucial information gathered throughout the experiment at a glance. The ranks that appear more frequently in this scheme are more opaque than the ranks that appear less frequently, which are lighter. Black is the darkest colour or the colour of the most commonly occurring ranks, while white is the lightest, showing that no ranks have been recorded.

\autoref{f1} and \autoref{f2} show the sensitivity analysis of TODIM and CPT-TODIM, respectively. We can infer the following points from the figures:

\begin{itemize}

    \item With a change in the weights of the criteria, CA (Canada) may get first or second place more frequently than third with TODIM. However, in the instance of CPT-TODIM, CA commonly ranked first the most.

    \item In the instance of FR (France), the rank typically oscillates between first and second only when the criterion weights for TODIM are changed. With CPT-TODIM, however, FR came in second place for most weight values.

    \item In case of DE (Deutschland or Germany), we can observe that third rank was the most common occurance for TODIM, while oscillation between third and fourth rank can be noticed in case of CPT-TODIM. 

    \item Sixth and seventh rank were observed the most for both TODIM and CPT-TODIM in case of IT (Italy) and JP (Japan), whereas other ranks occurred somewhat more in CPT-TODIM.

    \item With TODIM, the second and fourth rank occur the most frequently in the UK (United Kingdom), followed by the first and fifth rank, which occur less frequently. In the instance of CPT-TODIM, we can observe that the first and fourth rank appear the most frequent.

    \item Finally, in case of the United States of America (USA), we can observe that the first and fifth positions are the most common, followed by the fourth position for TODIM. However, in the instance of CPT-TODIM, only the fifth rank appears to be the most common.

\end{itemize}

The presence of the lighter tinted dots shows that, while particular ranks are less common, they may occur for some weights of the criterion. In terms of sensitivity to changes in criterion weight, the approaches TODIM and CPT-TODIM exhibit comparable behaviour, with CPT-TODIM exhibiting essentially no evidence of improvement. Also, the United Kingdom (UK) and the United States (USA) demonstrated a greater diversity of ranking occurrences, which is marginally worse with CPT-TODIM.
\begin{figure}[H]
\includegraphics[width=.6\textwidth]{TODIM-SA-WEIGHT}
\caption{Sensitivity Analysis of TODIM}\label{f1}
\end{figure}
\begin{figure}[!hbt]
\includegraphics[width=.6\textwidth]{CPT-TODIM-SA-weight}
\caption{Sensitivity Analysis of CPT-TODIM}\label{f2}
\end{figure}
\subsection{Rank Reversal}
Rank reversal is a phenomenon that occurs when a decision maker is selecting an alternative from a dataset and is given with additional options that were not examined when the selection process began \citep{aires2018rank}. In this article, we observed rank reversal by removing one country from the list, for all countries. We did not replace G7 countries with any other nation since the President of the European Council represents the G7 countries officially. As a result, there is little to no chance that decision-makers will attempt to transform any listed countries.

We discovered when CA is removed, the alternatives UK and USA switch positions for TODIM. When DE is removed, CA and FR switch places for both TODIM and CPT-TODIM. CPT-TODIM has a slightly lower rank reversal problem than TODIM. However, compared to other MCDM algorithms, both TODIM and CPT-TODIM display minimal rank reversals of alternatives.

\section{Concluding Remarks}

\subsection{Implementations of Policy}

The results in this paper provide a comparative analysis between TODIM and CPT-TODIM. According to our findings for both the algorithms, France outperformed all other nations. Therefore, we can also say that by ranking first France shows it has applied social sustainability principles in several social areas, such as employment rate, household wage, income fairness and inequity. As stated by \cite{torkayesh2021comparative} in 2015, France was one of the first countries to offer to join the Paris Climate Agreement and contribute to decrease \(CO_2\) emissions. This agreement boosted renewable energy initiatives, resulting in many job possibilities. France is followed by Canada, which is likewise regarded as one of the top leading countries to apply social sustainability requirements. In recent decades, they have concentrated on establishing equal pay for men and women for equal labour values, as well as a specific fund for maternity, newborn, and child health care.

\subsection{Comparative analysis of TODIM and CPT-TODIM}
The suggested decision-making paradigm is built in two steps. In the first phase, the weights of criteria are determined using the rank sum weight technique, and decision makers are free to rank the criteria as they see fit. In the second step, TODIM and CPT-TODIM, as ranking MCDM models, are used to compare seven G7 nations using real data from the OECD online database while considering the weighted criteria specified in step one. The findings produced in this study are comparable to those of the integrated CoCoSo algorithm researched by \cite{torkayesh2021comparative} however, both our techniques are reliable and simple enough for decision makers to grasp. The TODIM algorithm is a valuable tool for simulating the unreasonable elements of DMs, but it cannot capture the entire psychological states of DMs as described in CPT-TODIM \cite{tian2019extended}. However, we showed in this article using rigorous sensitivity analysis that TODIM and CPT-TODIM are closely linked and their behaviour is not improved on a broad scale in the later algorithm. TODIM places a great deal of importance on the pair-wise comparisons of all alternatives regarding all criteria. The evaluation of gains and losses takes up most of the computation time. CPT-TODIM tries to return some significance to the weights as specified by the preferences of the DMs. It does so by transforming the weights according to the parameters \(\alpha\) and \(\beta\). The computation of CPT-TODIM is more complicated than TODIM and both provide the same rankings, however the former is superior in terms of sensitivity of change in criteria weight and rank reversal.

Google Trends is an online tool that examines the popularity of top Google Search queries across several countries and languages. According to their analytics, if we compare the keywords of various popular MCDM approaches such as TOPSIS, VIKOR, CoCoSo and COPRAS with TODIM, within 18 years TODIM is the least popular out of all \citep{GTrendsMCDMComp}. However, we showed in this study that when the criterion weights are shuffled, the most frequent ranking of G7 nations for both TODIM and CPT-TODIM does not vary much over 1000 distinct shufflings. We know TODIM is less susceptible to changes in the weight of criterion and changes in parametric values as studied by \cite{prabjot2021todim}. Hence, using an improved version of TODIM that is CPT-TODIM, is strongly recommended, especially when the computations are not complicated, the results are more dependable, and cumulative prospect theory implementations are more important for decision makers. More research may be done to remove the rank reversal problem from this technique and to enhance the behaviour of the algorithm when the criterion weights are changed. To do all the computations in this article, including the sensitivity analysis, we used Python, a very popular high-level programming language. The run time complexity of both the methods is similar. However, CPT-TODIM has an additional \(O(nm^2)\) loop. Here \(n\) is the number of alternatives and \(m\) is the number of criteria. The GitHub repository where we have shared the code for future reference can be found here: \href{https://github.com/hungrybluedev/social-sustainability}{https://github.com/hungrybluedev/social-sustainability}.

%\begin{acknowledgement}[title={Acknowledgments}]
%We would like to thank the academic members of the Department of Mathematics, Birla Institute of Technology, Mesra, for their assistance and introducing us to multi-criteria decision making.
%\end{acknowledgement}

\begin{funding}
No funding was received for this research work.
\end{funding}

\bibliographystyle{infor}
\bibliography{biblio}
\vfill
\begin{biography}\label{bio1}
\author{V. Dutta} is a final year student pursuing her Integrated MSc. degree in Mathematics and Computing at Birla Institute of Technology, Mesra. She has contributed to research papers published in Mathematical Problems in Engineering and Complexity. Her research interests include mathematical ecology, biological mathematics, mathematical modelling, sustainability and multi-criteria decision making (MCDM) algorithms.
\end{biography}

\begin{biography}\label{bio2}
\author{S. Haldar} is a final year student pursuing his Integrated MSc. in Mathematics and Computing at Birla Institute of Technology, Mesra. He is a developer for the Open Source V programming language (https://vlang.io), and is responsible for the official blog posts (https://blog.vlang.io). He also reorganised and maintains the random number module - \texttt{rand}, as well as other mathematical modules like \texttt{math.frac}, \texttt{math.big}, etc. His contributions to research include papers in Mathematical Problems in Engineering and Complexity. His research interests include algorithm development and analysis, random number generation and multi-criteria decision making (MCDM) algorithms.
\end{biography}

\end{document}

